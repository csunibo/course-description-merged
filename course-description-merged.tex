\documentclass{article}
\usepackage{graphicx} % Required for inserting images
\usepackage{geometry}
\usepackage{hyperref}

\geometry{left=2cm, top=1.5cm, right=2cm, bottom=1.5cm}

\title{Course Descriptions}
\author{\textit{Your name}}

\makeatletter
\def\@seccntformat#1{%
  \expandafter\ifx\csname c@#1\endcsname\c@subsection\else
  \csname the#1\endcsname\quad
  \fi}
\makeatother

\begin{document}

\maketitle

\section{First Year}

\subsection{11925 - Computer Architecture}

Organisation of computer systems. Binary systems. Elements of Boolean algebra, Logic Gates, Combinational circuits, Sequential circuits. Memory, CPU and Bus. The ISA level and assembly programming. The operating systems.


\url{https://www.unibo.it/en/teaching/course-unit-catalogue/course-unit/2020/350960}

\subsection{93283 - Logics for Informatic (9 ECTS)}

FIRST PART: LOGICS

\begin{itemize}
    \item Paradoxes and their resolution. Applications of paradoxes to obtain negative results in computer science.
    \item Introduction to axiomatic set theory: ZF, relations, functions, quotients, cardinality. Cantor's theorem.
    \item Propositional languages: syntax and semantics. Satisfiability and semantic equivalence. Syntactical methods: propositional resolution and natural deduction. Soundness and completeness.
    \item First order languages. Predicates, terms, quantifiers. Syntax: free and bound variables. Interpretations. Semantics for a predicative language. Satisfiability and semantic equivalence.
    \item BNF to define grammars. Structural induction and recursion.
    \item Syntactical methods for first order. Natural deduction. Soundnes theorem. Completeness theorem. Compactness theorem.
\end{itemize}

SECOND PART: ALGEBRA

\begin{itemize}
    \item Generalization, abstraction, instantiation in mathematics and computer science. Higher order functions.
    \item Introduciton to algebra: semigroups, monoids, groups and their applications to generic programming.
\end{itemize}

\url{https://www.unibo.it/en/teaching/course-unit-catalogue/course-unit/2020/455095}

\subsection{00819 - Programming}

Introduction to Programming with C++.
Imperative programming in C++: algorithms and programs, data types, assignment, input / output, conditional, iteration, functions, recursion and recursive functions, vectors, records, memory allocation, dynamic data structures (lists, queues, trees)
Object-oriented programming in C++: classes, methods, overloading, inheritance.
Use of a development environment.

\url{https://www.unibo.it/en/teaching/course-unit-catalogue/course-unit/2020/320574}

\subsection{58414 - Algebra and Geometry}

\begin{itemize}
\item Linear Systems

\item Matrices

\item Gaussian elimination

\item Real vector spaces and subspaces

\item Linear maps

\item Eigenvalues and eigenvectors

\item Diagonalization of matrices

\item Modular arithmetic, congruence classes modulo n

\end{itemize}

\url{https://www.unibo.it/en/teaching/course-unit-catalogue/course-unit/2020/366975}

\subsection{37635 - Data Structure and Algorithms}

Data structures. Arrays, records, lists, stacks, queues. Trees. Tree visits (preorder, inorder, postorder). Sets. Dictionaries. Binary search. Hash tables. Priority queues. Heaps. Heapsort. Balanced search trees. MFSET. Graphs. DFS and BFS. Design and analysis of algorithms. Computational complexity. Order of growth. Recurrence equations. Lower bounds. Design techniques: divide-\&-conquer, backtrack, greedy, local search, dynamic programming. Sorting: Mergesort, Quicksort, Shellsort. Algorithms on graphs: Minimum Spanning Tree (Prim, Kruskal), Shortest Paths (Bellman-Ford, Dijkstra, Floyd-Wharshall). Complexity. The P and NP classes. NP-completeness.

There is also a module of laoratory in which data structures are implemented and used, and the Object-Oriented paradigm as well as some Java notions are introduced.


\url{https://www.unibo.it/en/teaching/course-unit-catalogue/course-unit/2020/350957}

\subsection{00013 - Mathematical Analysis}

he number sets: N,Z,Q,R.
The induction principle.
Sequences of real numbers.

Differential calculus for functions of one real variable.
Exponential and logarithmic functions. Trigonometric funtions.
Limits. Continuity (local and global properties).
Derivatives, monotony. Local maxima and local minima.
Infinite and infinitesimal asymptotics. Taylor's formula.

Integral calculus for functions of one real variable: primitive and integral, techniques of integration (by parts, by substitution), integral of rational functions. Generalized integrals.

An introduction to differential calculus for function of several variables. Continuity and derivatives for functions of several real variables.
Integral calculus for functions of two real variables.

\url{https://www.unibo.it/en/teaching/course-unit-catalogue/course-unit/2020/320573}

\section{Second Year}

\subsection{02023 - Numerical Computing}

\begin{itemize}
    \item Floating point numbers and finite arithmetics.

    \item Direct and iterative numerical methods for the solution of linear systems. The least squares formulation.

    \item Data and functions interpolation.

    \item Minimization of functions in one and more variables. Numerical algorithms for roots finding. Descent methods for multivariable functions minimization.

    \item Introduction to inverse problems in imaging: denoise, deblur, super-resolution, image reconstruction from projections.

    \item Exercises in Python.
\end{itemize}

\url{https://www.unibo.it/en/teaching/course-unit-catalogue/course-unit/2021/320581}

\subsection{04642 - Probability Calculus and Statistics}

\begin{itemize}
    \item Mathematical model of a random experiment: sample space, events, axioms of probability and their consequences.

    \item Conditional probability and independence: chain rule, total probability rule and Bayes' rule.

    \item Combinatorics and discrete uniform probability spaces.

    \item Random variables

    \item Distribution (or law) and cumulative distribution function.
    Discrete and (absolutely) continuous random variables: discrete and continuous probability density functions.
    \item Expected value and variance.
    \item Common probability distributions: Bernoulli, binomial, Poisson, discrete uniform, continuous uniform, exponential, normal (or Gaussian). 

    \item Random vectors

    \item Joint law, marginal laws, joint cumulative distribution function, independence of random variables, covariance.
    Discrete random vectors: joint discrete and marginal discrete probability density functions.

    \item Descriptive statistics: population and sample, types of data, frequencies, tabular and graphical representations; measures of central tendency, measures of variability.

    \item Bivariate data: joint frequencies and two-way tables; scatter plot; covariance and linear correlation coefficient; method of least squares and linear regression.

    \item Limit theorems

    \item Sequence of i.i.d. random variables.
    \item Law of large numbers: Chebyshev's inequality, Monte Carlo method.
    \item Central limit theorem.

    \item Discrete-time Markov chains: transition matrix, directed graph representation, n-step transition probability, communication classes, invariant distribution. 

\end{itemize}

\url{https://www.unibo.it/en/teaching/course-unit-catalogue/course-unit/2021/320583}

\subsection{88566 - Web Technology (9 ECTS)}


\begin{itemize}
    \item Fundamentals: VII level protocols, character encodings, standard bodies
\item Basic web technologies: HTTP, URI, HTML, CSS, XML
\item Server-side technologies for web applications: php, python, NodeJs
\item Client-side technologies for web applications: JavaScript, Ajax, JSON, JavaScript frameworks.
\item Component-based web programming: Angular, React, Vue.
\item Introduction to some technologies of Semantic Web: RDF, OWL, SPARQL, ontologies.
\item User Experience Design for web sites: the Garrett model
\end{itemize}

\url{https://www.unibo.it/en/teaching/course-unit-catalogue/course-unit/2021/436428}

\subsection{04138 - Programming Languages}

The evolution of programming languages. From assembly to higher level languages. Abstract machines, intepreters and compilers. Description of a programming language: syntax, semantics, pragmatics and implementation. Syntax (BNF). Structured Operational Semantics (SOS). Regular grammars, regular expressions, and finite automata: equivalences and principal theorems (e.g., pumping lemma). Design of lexical analysers. Lex. Context free grammars and push-down automata: equivalences and principal theorems (e.g., pumping theorem). Deterministic context free grammars: algorithms for parsing; grammars LL(1), LR(0), SLR, LR(1), LALR(1). YACC.

Environment, scoping rules and their implementation. Stack of the activation records; heap. Memory management: garbage collection. Sequence control, procedures, recursion. Types and type checking. Parameters and parameter passing: by value, by reference, by result, by name. Functional parameters; closures. Exceptions. The object-oriented paradigm: classes and objects, initialization, inheritance and late-binding. Subtyping is not inheritance. The logical paradigm. The functional paradigm (Scala). The concurrent paradigm and service-oriented computing (Jolie).

\url{https://www.unibo.it/en/teaching/course-unit-catalogue/course-unit/2021/320579}

\subsection{93315 - Computer Networks (12 ECTS)}
\begin{itemize}
\item Foundation topics: definitions, history and development of computer networks.
\item Topologies, network resources, and logical channels.
\item Computer Network performances: indexes and their meaning in different application contexts.
\item Circuit-switched and packet-switched networks.
\item Network communication protocols.
\item Network architectures: HW and SW.
\item Network Service architectures: Client/server, Peer to peer, hybrid.
\item ISO OSI Reference Model.
\item Physical layer: transmission medium, signals, encoding/decoding.
\item Data Link layer: communication channels, Medium Access Control techniques, MAC addressing, Reliable communication, Error detection and correction.
\item Local Area Network technologies: hub, repeater, bridge, switch. LAN connectivity.
\item LAN topologies and links.
\item Virtual channels (MPLS) and virtual networks (VLAN).
\item Network Layer: IPv4 protocol and addressing. IPv6. Domains and hierarchical adressing. Subnetting and supernetting, IPv4 network classes, CIDR, IP configuration. Network Address Translation (NAT). SDN e OpenFlow. ICMP. ARP e RARP. DHCP.
\item Design of network and subnetworks in IP domains.
\item Management and configuration of LANs (SNMP).
\item Troubleshooting and analysis of network performance and issues.
\item Networks of networks and inter-networking. Forwarding and routing \item IP (local and ISP-based - interdomain). Router.
\item Multicasting.
\item Transport layer: Transmission Control Protocol (TCP), performance of end-to-end communications, Congestion control. Flow control.
\item Sockets and socket programming (examples) with UDP/TCP.
\item Inter-process communications over Internet.
\item Session and Presentation layers.
\item Application layer: examples of protocols and services at the \item application layer. SMTP (email), http (WWW), DNS, streaming video, gaming, P2P, VoIP.
\item Quality of service. Real Time communication.

\item Security of communication networks. Privacy, crittography, integrity and digital signature. Secure transport layer (TCP): SSL. Secure network layer. IPsec. Virtual Private Networks (VPN). Firewalls and Intrusion Detection.
\end{itemize}

\url{https://www.unibo.it/en/teaching/course-unit-catalogue/course-unit/2021/455456}

\subsection{08574 - Operating Systems}

\begin{itemize}
    \item Operating Systems: definition and history
    \item Concurrent Programming
    \item Structure of an O.S.
    \item Scheduling
    \item Resource Management
    \item Main Memory Management
    \item Secondary Memory Management
    \item File Systems
    \item Security of Operating Systems
    \item the C language
    \item Programming Tools
    \item Shell scripting
    \item The Python Language

\end{itemize}

\url{https://www.unibo.it/en/teaching/course-unit-catalogue/course-unit/2021/320578}

\section{Third Year}

\subsection{90107 - DATA BASE (9 ECTS)}

\begin{itemize}
    \item Databases
    \item Relational data model
    \item Relational algebra and calculus
    \item SQL
    \item Database design methodology
    \item ER data model and quality verification
    \item Laboratory: notions about the architecture of a DBMS, indexes, transactions and design examples

\end{itemize}

\url{https://www.unibo.it/en/didattica/insegnamenti/insegnamento/2022/443720}

\subsection{90106 - Software Engineering (9 ECTS)}

Software products and their development •Software lifecycle •Agile software development methods •Agile Scrum •Requirement engineering •Design patterns •Modeling software with UML •Software development tools •Project Management for software systems • Controlling and measuring software quality • Software maintenance •Configuration management

\url{https://www.unibo.it/en/teaching/course-unit-catalogue/course-unit/2022/443719}

\subsection{41169 - Theoretical Computer Science (6 Credits)}

The student will learn the main notions and results of Computability and Complexity Theory. At the end of course the student will be aware of the theoretical and practical limits of computation, and will be able to use and apply methodologies and techniques typical of formal methods to the study and solution of a wide range of algorithmic problems

\url{https://www.unibo.it/en/teaching/course-unit-catalogue/course-unit/2022/455107}

\subsection{93319 - Introduction to machine leaning (6 Credits)}

The first part of the course provides a general introduction to the field of machine learning, in its typical forms: supervised, unsupervised, with reinforcement. Traditional topics such as decision tree learning, logistic regression, Bayesian networks and Support Vector Machines will be covered.

The second part of the course is focused on Neural Networks, and their typical learning mechanism: the backpropagation algortihm. We shall discuss the main types of neural nets: feed forward, convolutional, recurrent, and their practical applications. We shall also investigate techniques to visualize the effect of hidden units (tightly related to deep dreams and inceptionism) as well as several generative approaches comprising Generative Adversarial Networks. Thematics relative to Object Detection and Semantic Segmentation will be also briefly discussed too.

\url{https://www.unibo.it/en/teaching/course-unit-catalogue/course-unit/2022/455112}

\subsection{93466 - CYBERSECURITY}

The objective of the course is to present the theory, mechanisms, techniques and tools that are effective in increasing the security of a computer system. At the end of the course, the student will be familiar with the mathematical foundations of modern cryptography, authentication, authorization and access control mechanisms that are suitable for achieving confidentiality, integrity and availability of computer systems. The student will also acquire the knowledge necessary to assess the potentials and limitations of current technologies.

\url{https://www.unibo.it/en/teaching/course-unit-catalogue?search=True&annoAccademico=2022&codiceCorso=8009&single=True&codiceMateria=93466}

\section{Elective Courses}

The classes I took during my second semester of my third year are listed below. 



\end{document}
